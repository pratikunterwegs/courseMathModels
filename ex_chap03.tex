\hypertarget{problem-3.16}{%
\section{Problem 3.16}\label{problem-3.16}}

\textbf{Problem statement:} Defining the fitnesses of three diploid
phenotypes as follows, \(W_{AA} = 1 + s\), \(W_{Aa} = 1 + hs\), and
\(W_{aa} = 1\), we want to show that the recursion equation for the
frequency \(p\) of allele \(A\) at time \(t+1\):

\begin{equation} p(t+1) = {p(t)}^2 \frac{W_{AA}}{\overline{W}} + \frac{1}{2} (2 p(t) q(t) \frac{W_{Aa}}{\overline{W}}) \end{equation}

can be used to obtain the following continuous-time differential
equation:

\begin{equation} \frac{dp}{dt} = sp(1-p)(p + h(1 - 2p)) \end{equation}

and that this is equivalent to:

\begin{equation} \frac{dp}{dt} = p(1-p)(p(W_{AA} - W_{Aa}) + (1-p)(W_{Aa} - W_{aa})) \end{equation}

\textbf{Part 1:}

\begin{enumerate}
\def\labelenumi{\arabic{enumi}.}
\tightlist
\item
  We begin by writing the difference equation form of (1):
\end{enumerate}

\begin{equation} \Delta p  = p(t+1) - p(t) \end{equation}

which works out to:

\begin{equation} \Delta p = {p(t)}^2 \frac{W_{AA}}{\overline{W}} + \frac{1}{2} (2 p(t) q(t) \frac{W_{Aa}}{\overline{W}}) - p(t) \end{equation}

\begin{enumerate}
\def\labelenumi{\arabic{enumi}.}
\setcounter{enumi}{1}
\tightlist
\item
  Replacing \(p(t)\) and \(q(t)\) with \(p\) and \(q\), and \(W_{AA}\),
  \(W_{Aa}\), and \(W_{aa}\) by the values above, and by fraction
  subtraction:
\end{enumerate}

\begin{equation} \Delta p = \frac{p^2(1+s) + pq(1+hs) - p(\overline{W})}{\overline{W}} \end{equation}.

\begin{enumerate}
\def\labelenumi{\arabic{enumi}.}
\setcounter{enumi}{2}
\tightlist
\item
  Replacing \(q\) with \(1-p\), and expanding \(\overline{W}\) in the
  numerator:
\end{enumerate}

\begin{equation} \Delta p = \frac{p^2(1+s) + p(1-p)(1+hs) - p(p^2(1+s) + (1-p)^2 + 2p(1-p)(1+hs))}{\overline{W}} \end{equation}

\begin{enumerate}
\def\labelenumi{\arabic{enumi}.}
\setcounter{enumi}{3}
\tightlist
\item
  Factorising by \(p(1-p)\) as follows, first by \(p\), and then by
  \(1-p\):
\end{enumerate}

\begin{equation} \Delta p = \frac{p[p(1+s)(1-p) + (1-p)(1 + hs - 1 + p - 2p(1+hs)]}{\overline{W}} \end{equation}

\begin{equation} \Delta p = \frac{p(1-p)[p(1+s) + ((1+hs)(1-2p)) - (1-p)]}{\overline{W}} \end{equation}

which reduces the numerator on expansion to:

\begin{equation} \Delta p = \frac{p(1-p)[s(p+h - 2ph)]}{\overline{W}} \end{equation}

and finally upon exapanding \(\overline{W}\) in the denominator:

\begin{equation} \Delta p = \frac{sp(1-p)[p+h(1-2p)]}{1 + s(p^2 + 2ph - 2p^2h)} \end{equation}

\begin{enumerate}
\def\labelenumi{\arabic{enumi}.}
\setcounter{enumi}{4}
\tightlist
\item
  Over small timesteps \(\Delta t\), the rate of change of \(p\) is:
\end{enumerate}

\begin{equation} \frac{\Delta p}{\Delta t} = \lim \limits_{\Delta t \to 0} \frac{s \Delta t p(1-p)[p+h(1-2p)]}{1 + s \Delta t (p^2 + 2ph - 2p^2h)} \cdot \frac{1}{\Delta t} \end{equation}

and at the limit of \(\Delta t = 0\), we obtain equation (2):

\begin{equation} \frac{dp}{dt} = sp(1-p)[p + h(1-2p)] \end{equation}

\textbf{Part 2:}

\begin{enumerate}
\def\labelenumi{\arabic{enumi}.}
\setcounter{enumi}{5}
\tightlist
\item
  Working from equation (3), and substituting the values of the
  phenotype fitnesses:
\end{enumerate}

\begin{equation} \frac{dp}{dt} = p(1-p)(p(1+s - 1-hs) + (1-p)(1+hs-1)) \end{equation}

\begin{equation} p(1-p)(ps - 2phs + hs) \end{equation}

yields the continuous-time equation (12):

\begin{equation} sp(1-p)(p + h(1-2p)) \end{equation}

\begin{center}\rule{0.5\linewidth}{\linethickness}\end{center}
